\chapter{Advanced Topics}

\section{Running arbitrary R functions}

RMINC has the capacity to run arbitrary R functions at ever voxel of a
set of files. This comes in quite handy when there are no easily
wrapped functions that exist in RMINC but there is some existing R
module you would like to try out. Some words of caution are in order,
however:

\begin{itemize}
\item Running arbitrary functions involves writing your own small R
  function to wrap the code you want to use, which is a bit ugly.
\item It is slow. Slower than molasses on a cold Georgia winter
  morning.
\end{itemize}

Here's an example of how it works:

\begin{Schunk}
\begin{Sinput}
> f <- function() {
+     return(tapply(x, gf$Genotype, mean))
+ }
> vs <- mincApply(gf$jacobians, quote(f()), mask = "small-mask.mnc")